With the development of column-based in-memory database systems the use of
array-based programming languages have shown to be promising for the execution
of database queries. In particular, they allow for an elegant representation of
query execution plans, that then can be further optimized using well
established techniques from compiler optimizations. However, it is not
straightforward to apply standard optimization techniques such as loop fusion
as database operators are best represented by advanced built-in functions that
can be quite complex. 
In this work, we apply a compiler approach to optimize SQL query execution
plans that are expressed in an array-based intermediate representation. We
analyze this code to determine shape properties of the data being
processed, and use a subsequent optimization phase to fuse multiple database
operators into single, compound operations, reducing the need for separate
computation and storage of intermediate values.  Experimental results on a
range of TPC-H queries show that our fusion technique is effective in
generating efficient code, improving query time over a baseline system.


%However, the database operators found in SQL queriessuch as loop fusion. They
%In this paper, we provide a systematic approach to apply loop fusion to
%array-based programming languages that are designed to support database
%queries. In-memory database systems (IMDBs) offer significant performance
%benefits to many applications in data analytics.
%With reduced I/O cost, however, good CPU performance is critical, but also
%difficult to achieve given the step-by-step nature of SQL execution.
%In this work we apply a compiler approach to optimize SQL queries in IMDBs.
%SQL execution plans are first translated to a custom, array-based intermediate
%representation.
%We analyze this code to determine shape properties of the data being processed,


% technique is effective in generating efficient code, improving query time by
% X\% over a baseline IMDB system.
