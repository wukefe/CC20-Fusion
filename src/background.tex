% \hanfeng{Working on this section.  Please come back later!}

In this section, we first present the principles of loop fusion techniques for
array-based optimization, and then provide background on HorseIR and its
optimization capabilities. 

\subsection{Loop Fusion}

Loop fusion is an effective technique to improve program performance as
loops usually dominate the execution.
In the context of array-based languages, loop fusion is applied during the code
generation phase, when array-based functions are translated into loops of the
target language, e.g., C. The goal of loop fusion is to minimize the number of
loops in the target program.
There are two kinds of loop fusion as follows.

1. \textit{Fusion for removing intermediate results.}
For example, assume the sequence of two element-wise functions
\texttt{T = A*B, R = T+C}, where T, A, B, R, C are vectors of the same length
with no memory overlap.
A naive code-generation would translate this line-by-line into two loops for
the two arithmetic functions, first multiplication, then addition, with an
intermediate result vector T.
However, the element-wise functions can be easily fused generating a single
loop without intermediate vector (i.e., with loop body \texttt{R[i]=A[i]*B[i]+C[i]}).

2. \textit{Fusion when sharing the same loop head.}
Generally, when two loops share the same loop head (but have different loop
body), they often can be fused into a single loop in order to reduce the
overhead of iterating over the two loops individually. However, the bodies of
the two loops might have dependencies that make such a merge impossible. 

%%% \begin{lstlisting}[style=CStyle]
%%% // temp[10]: a temporary array
%%% for(int i=0; i<10; i++){
%%%         temp[i] = C[i] * D[i];
%%% }
%%% for(int i=0; i<10; i++){
%%%         A[i] = B[i] + temp[i];
%%% }
%%% \end{lstlisting}

%%% \begin{lstlisting}[style=CStyle]
%%% for(int i=0; i<10; i++){
%%%         A[i] = B[i] + C[i] * D[i];
%%% }
%%% \end{lstlisting}

%%% \todo{Explain how challenging loop fusion is.}

Ideally, a smart compiler can optimize loops in steps:
(1) resolve the data dependency between variables inside loops;
(2) identify the use of temporary, intermediate result arrays; and
(3) fuse the for loops and avoid the intermediate result arrays whenever possible.

However, automatic loop fusion is challenging for
compilers~\cite{Kennedy01:FastFusion,Kennedy1993:LoopFusion}.
Instead, we believe it is much easier if the potential of loop fusion is
already performed looking at the high-level language
(i.e., in the array-based program) due to the higher level of abstraction.

\subsection{HorseIR: an Array-based IR for SQL Queries}

HorseIR \OldPaper was designed to represent execution plans for SQL queries
where data is represented in column-format and assumed to reside in
main-memory. HorseIR represents the columns of the database tables as vectors,
and uses lists for compound data. It provides dozens of data types to
accommodate the many data types found in database systems. Moreover, it supports a
large set of array-based built-in functions that  represent the arithmetic and
database-related operators that are commonly used in SQL queries.

Fig.~\ref{fig:motivation_example} shows a motivating example of queries
executing with HorseIR framework. Fig.~\ref{fig:motivation_sql} shows a SQL
query on the database table \texttt{store\_items}. The WHERE clause represents
a selection filtering only those records of the table that fulfill certain
conditions on the \texttt{item_date} column. The SELECT clause projects on
columns \texttt{item_price} and \texttt{item\_discount}, performing an
element-wise arithmetic function (multiplication) and then aggregation over the
result. 

% http://www.texample.net/tikz/examples/simple-flow-chart/
\tikzstyle{line}  = [draw, ->, >=stealth]
\tikzstyle{block} = [node distance=8mm]

\begin{figure*}[htbp]

\begin{subfigure}[b]{.6\columnwidth}
\begin{lstlisting}[style=SQLStyle, basicstyle=\footnotesize]
SELECT
  SUM(item_price * item_discount) AS saving
FROM
  store_items
WHERE
  item_date > 2010.09.01 AND
  item_date < 2010.09.30;
\end{lstlisting}
\caption{An SQL query example} \label{fig:motivation_sql}
\end{subfigure}
\hfill
\begin{subfigure}[b]{.7\columnwidth}
\begin{lstlisting}[language=HorseIR, basicstyle=\footnotesize, literate = {-}{-}1]
// ... load columns from table
(S0) t0:bool = @gt(c0,2010-09-01:date);
(S1) t1:bool = @lt(c0,2010-09-30:date);
(S2) t2:bool = @and(t0, t1);
(S3) t3:f64  = @compress(t2, c1);
(S4) t4:f64  = @compress(t2, c2);
(S5) t5:f64  = @mul(t3, t4);
(S6) t6:f64  = @sum(t5);
// ... return result as a table
\end{lstlisting}
\caption{Core part of HorseIR code for (a) } \label{fig:motivation_horseir}
\end{subfigure}
% \hfill
\hspace{0.07\columnwidth}
\begin{subfigure}[b]{.6\columnwidth}
\centering
% xleftmargin=.15\columnwidth
\begin{lstlisting}[style=CStyle, basicstyle=\footnotesize]
//... load columns c0,c1,c2
t6 = 0;
for(int i=0; i<n; i++){
  if(c0[i] > 20100901
     && c0[i] < 20100930){
    t6 += c1[i] * c2[i];
  }
}
//... return t6, a scalar
\end{lstlisting}
\caption{Optimized C Code generated from (b)} \label{fig:motivation_c}
\end{subfigure}
\caption{Translating an SQL query from (a) to the corresponding HorseIR code shown in (b) (with columns: c0 (item_date), c1 (item_price), and c2 (item_discount)), and generating optimized C code in (c).} \label{fig:motivation_example}
\end{figure*}


Such a query is first translated into a query execution plan. The previous
work~\OldPaper uses the advanced query optimizer of the HyPer database
system~\cite{Neumann2011:HyPer} to generate high-performance execution plans.
Then, the execution plan is translated into HorseIR using both standard array
language and database-specific built-in functions. By not directly translating
queries into HorseIR but instead translating the optimized execution plans, the
approach leverages the extensive SQL-specific query optimization experience of
the database community.

Fig.~\ref{fig:motivation_horseir} shows the resulting HorseIR program without
optimization and many intermediate results. From there, the previous
work~\OldPaper proposes optimizations based on element-wise fusion and fusion
with patterns. For example, (S0) and (S1) in the HorseIR program are easily
fused into one statement, as \texttt{@gt} and \texttt{@lt} are element-wise
functions. Since there is a large set of such functions, this approach has
shown to be very beneficial. 
Moreover, generating parallel code for element-wise functions is easy as they
are dependence-free.


However, once functions are more complex, such as the \texttt{@compress}
functions of lines (S3) and (S4) in the example, fusion is no longer
straightforward, as these high-level built-in functions contain data-dependent
loops implicitly. 
Thus, fusion across such functions is not supported by the previous
work~\OldPaper which instead generates separate loops for them.
This can become a severe performance bottleneck, especially when the
intermediate results are barely reused. 

The boolean selection function \texttt{@compress} is common in SQL queries
because of the WHERE clause.
In the example, \texttt{t2} is a boolean vector indicating the indices of the
records that fulfill the filtering condition of the query. In (S3) and (S4)
@compress retrieves the corresponding element values of the
\texttt{item\_price} and \texttt{item\_discount} columns.
However, the intermediate results of \texttt{t3} and \texttt{t4} can be avoided
because the retrieval can be fused with the subsequent arithmetic and
aggregation functions. In fact, for the given example, it is possible to fuse and
generate a single loop as shown in Fig.~\ref{fig:motivation_example} (c).

In this paper, we show how fusion \textit{across} built-in functions can be
done automatically and systematically by using a principled shape and
conformability analysis. Importantly, although previous work~\OldPaper also fuses
operations, it is based on simplistic element-wise fusion and pattern rules. Using
patterns is quite challenging as it requires the expertise of SQL execution plans.
Furthermore, there exist many different types of queries, and thus potentially many
different patterns. It is not clear, whether the patterns presented in \OldPaper are
exhaustive. 