% http://www.texample.net/tikz/examples/simple-flow-chart/
\tikzstyle{line}  = [draw, ->, >=stealth]
\tikzstyle{block} = [node distance=8mm]

\begin{figure*}[htbp]

\begin{subfigure}[b]{.6\columnwidth}
\begin{lstlisting}[style=SQLStyle, basicstyle=\footnotesize]
SELECT
  SUM(item_price * item_discount) AS saving
FROM
  store_items
WHERE
  item_date > 2010.09.01 AND
  item_date < 2010.09.30;
\end{lstlisting}
\caption{An SQL query example} \label{fig:motivation_sql}
\end{subfigure}
\hfill
\begin{subfigure}[b]{.7\columnwidth}
\begin{lstlisting}[language=HorseIR, basicstyle=\footnotesize, literate = {-}{-}1]
// ... load columns from table
(S0) t0:bool = @gt(c0,2010-09-01:date);
(S1) t1:bool = @lt(c0,2010-09-30:date);
(S2) t2:bool = @and(t0, t1);
(S3) t3:f64  = @compress(t2, c1);
(S4) t4:f64  = @compress(t2, c2);
(S5) t5:f64  = @mul(t3, t4);
(S6) t6:f64  = @sum(t5);
// ... return result as a table
\end{lstlisting}
\caption{Core part of HorseIR code for (a) } \label{fig:motivation_horseir}
\end{subfigure}
% \hfill
\hspace{0.07\columnwidth}
\begin{subfigure}[b]{.6\columnwidth}
\centering
% xleftmargin=.15\columnwidth
\begin{lstlisting}[style=CStyle, basicstyle=\footnotesize]
//... load columns c0,c1,c2
t6 = 0;
for(int i=0; i<n; i++){
  if(c0[i] > 20100901
     && c0[i] < 20100930){
    t6 += c1[i] * c2[i];
  }
}
//... return t6, a scalar
\end{lstlisting}
\caption{Optimized C Code generated from (b)} \label{fig:motivation_c}
\end{subfigure}
\caption{Translating an SQL query from (a) to the corresponding HorseIR code shown in (b) (with columns: c0 (item_date), c1 (item_price), and c2 (item_discount)), and generating optimized C code in (c).} \label{fig:motivation_example}
\end{figure*}
