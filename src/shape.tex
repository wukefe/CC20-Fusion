Shape information is key to identify fusible sections of built-in functions.
%Previous work explored simple fusion opportunities for element-wise functions and used numerous pre-defined patterns, but was unable to automatically handle more complex functions common to SQL queries (e.g. boolean selection - \texttt{@compress}).
In this section we:
(1) introduce our shape abstraction;
(2) categorize built-in functions by shape behaviour; and
(3) describe propagation rules for each category.

\subsection{Shape Abstraction}

Shapes describe the in-memory layout of data. HorseIR has two important shapes
used in queries: vectors and lists.

\subsubsection{Vector}

A vector shape describes a fixed-length one-dimensional array of homogeneous data. It
is therefore characterized based on the number of elements as shown in \refTable{table:shapedef}.

\begin{table}[htbp]
\centering
\caption{Definitions of vector shapes.} \label{table:shapedef}
\begin{small}
\begin{tabular}{|c|l|}
\hline
Shape       & \multicolumn{1}{c|}{Description}  \\ \hline
\shapeS     & Vector of constant size 1 (i.e. scalar)  \\ \hline
\shapeV{c}  & Vector of constant size \texttt{c} where c $\neq$ 1  \\ \hline
\shapeV{d}  & Vector of unknown static size (unique ID \texttt{d})  \\ \hline
\shapeVS{a} & Vector from boolean selection \texttt{a}  \\ \hline
\end{tabular}
\end{small}
\end{table}

The number of elements may either be a compile time constant, or a dynamic value
only known at runtime. We include a separate shape for scalar data as some built-in
functions exhibit specialized behaviour depending on the exact size.

Dynamic vector shapes describe objects that depend on runtime properties. Our system
assigns a unique ID to such vectors, so two vectors with the same dynamic ID have the
same shape. A specialized dynamic shape, \shapeVS{a}, describes the output result
of the boolean selection function \texttt{@compress} with boolean mask \texttt{a}. Our
code generation strategy uses the this shape for further fusion and
avoids storing intermediate results.

\subsubsection{List}

A list shape is composite, storing heterogeneous data in an ordered group of
\textit{cells}. Each cell has its own shape, either a vector or a nested list.
\begin{small}
\begin{verbatim}
list_shape ::= { cell_shape };
cell_shape ::= list_shape | vector_shape;
\end{verbatim}
\end{small}
We denote a list shape as \texttt{list<$L_0,L_1,...,L_{n-1}$>}, where $L_i$
is the shape of cell $i$ for \texttt{n} cells. Note that for SQL queries,
list cells are always vectors, as they typically represent collections of columns
or row indices.

% (1) \shapeL{n}{1} means a list which has \texttt{n} vectors and each vector has the length 1;
% (2) \shapeL{n}{d} means a list which has \texttt{n} vectors and and a unique ID
% \texttt{d} to represent all vectors not agree on the length 1; and
% (3) an unknown shape.

%% ALEX: This doesn't seem right here

%% Having lists in HorseIR programs is common due to several database related
%% operations, such as groupby and join.
%% \refFig{fig:list_example} presents an example query with a groupby operation.
%% Given an example table \texttt{Person} in \refFig{fig:list_table}, which has
%% (1) the columns \texttt{firstname} and \texttt{lastname} store the first and
%% last names of persons respectively; and
%% (2) the column \texttt{age} stores the ages of persons.
%% \refFig{fig:list_query} shows a query on the table that answers a question:
%% \textit{What is the average age of persons who share the same last name?}

%% \begin{figure}[htbp]
%% \begin{subfigure}[b]{.5\columnwidth}
%% \begin{tabular}{|c|c|c|}
%% \hline
%% firstname & lastname & age \\ \hline
%% HF1       & Chen     & 18 \\ \hline
%% Alex      & Krolik   & 23 \\ \hline
%% HF2       & Chen     & 32 \\ \hline
%% \end{tabular}
%% \caption{An example table \texttt{Person}.} \label{fig:list_table}
%% \end{subfigure}
%% \hspace{3mm}
%% \begin{subfigure}[b]{.4\columnwidth}
%% \begin{lstlisting}[style=SQLStyle]
%% SELECT
%%     AVG(age)
%% FROM
%%     Person
%% GROUP BY
%%     lastname;
%% \end{lstlisting}
%% \caption{A simple query.} \label{fig:list_query}
%% \end{subfigure}
%% \caption{An example query with a groupby operation.}\label{fig:list_example}
%% \end{figure}

%% \begin{figure}[htbp]
%% \begin{lstlisting}[language=HorseIR, frame=single]
%% // columns
%% firstname:str = ("HF1","Alex","HF2"):str;
%% lastname:str  = ("Chen","Krolik","Chen"):str;
%% age:i32       = (18, 23, 32):i32;
%% // compute
%% g:dict<i32,i32> = @group(lastname);
%% v:list<i32>   = @values(g);
%% t0:list<i32>  = @each_right(@index,age,v);
%% t1:list<i32>  = @each(@avg,t0);
%% avg:i32 = @raze(t1);
%% \end{lstlisting}
%% \caption{HorseIR code for \refFig{fig:list_query}} \label{fig:list_horseir}
%% \end{figure}


%% \todo{
%%     explain this example clearly, list and cells, (explain row id ...), group
%%     by; fetch value from dictionary.
%% }
%% First, the column \texttt{lastname} is performanced with a groupby operation to
%% get a dictionary having the result of row IDs for keys and values stored in the
%% variable \texttt{g}.  Its value can be seen as a tuple with key and value parts:
%% \texttt{\textless(0,1):i32, ((0,2),1):i32\textgreater}.
%% Since the variable \texttt{v} fetches the value from the dictionary, it is a
%% vector and its value is \texttt{((0,2),1):i32}
%% After array indexing for each cell inside the value \texttt{v}, we can have a
%% new list with the value of \texttt{((18,32),23):i32} stored in the variable
%% \texttt{t0}. Clearly, we see that the two ages who share the same last names
%% %% are put together in the first cell.
%% Since the built-in function \texttt{avg} cannot work on a list directly, we need
%% a function \texttt{each} which takes a unary function on each cell.
%% Finally, a function \texttt{raze} destroys barriers between cells, merges all
%% integers into a vector, and returns it (i.e. (25,23):i32)

%% \begin{figure}
%% \begin{verbatim}
%% typedef struct InfoNode{
%%     Type type;
%%     struct ShapeNode *shape;
%%     struct InfoNode *cell;
%%     struct InfoNode *next;
%% } InfoNode;
%% \end{verbatim}
%% \caption{An InfoNode for storing type and shape information described in C language.}
%% \label{fig:list_struct}
%% \end{figure}

%% \headstep{Data structure.}
%% \refFig{fig:list_struct} shows the data structure for storing type and shape
%% information for our conformability analysis.  
%% As can be seen, a recursive data structure is used to preserve the type and
%% shape information for lists.  It is called \texttt{InfoNode}.  Its shape
%% information is stored in a shape node pointed by the field \texttt{shape}.
%% The shape node is designed to represent different shapes defined in
%% \refSec{SubSec:shape_abs_vector} and \refSec{SubSec:shape_abs_list}.

\subsection{Built-in Function Categories}\label{SubSec:groups}

Built-in functions are categorized based on their predefined shape behaviours.
This simplifies later analyses which depend on the shape behaviour and not
the exact operation. For example, element-wise binary functions \texttt{@plus}
and \texttt{@mul} share identical structure and may therefore share a single
set of shape propagation rules.
\begin{small}
\begin{Verbatim}[xleftmargin=.3\columnwidth]
a:i32 = @plus(A, B);
b:i32 = @mul(A, B);
\end{Verbatim}
\end{small}
There are two kinds of built-in functions in HorseIR: functions for vectors and
functions for lists. They can be categorized as follows:

\begin{description}
\item[Element-wise (E)]:
    unary and binary functions, including arithmetic, boolean, and math. They are frequently used to represent the operators found in the WHERE clause (selection) and the SELECT clause (projection);

\item[Reduction (R)]:
    reduction functions \texttt{@sum}, \texttt{@avg}, \texttt{@min}, and \texttt{@max}. Aggregation functions of SQL;

\item[Scan (S)]:
    boolean selection functions \texttt{@compress}. After the selection, they retrieve the relevant elements for the projection;

\item[Indexing (X)]:
    indexing function \texttt{@index};

\item[Special Boolean (B)]:
    functions that return a boolean vector without implicit data dependency, such as
    \texttt{@like} and \texttt{@member};

\item[Each (H)]:
    list functions \texttt{@each}, \texttt{@each_left}, \texttt{@each_right},
    \texttt{@each_item}, and \texttt{@raze}. They are often needed in SQL statements with a GROUP BY;

\item[Others (O)]:
    all other functions.
\end{description}
Groups can be extended as needed with additional built-in functions as
the language and libraries evolve. Each group is also associated with
an abbreviation that is used in the following sections.

\subsection{Shape Propagation Rules} \label{SubSec:propagation}

Shape propagation rules are defined for each category of vector functions,
list functions \texttt{@each*} and other functions.

\subsubsection{Vectors}

For vector functions, the return shape can either be:
(1) a parameter shape;
(2) a new vector shape;
(3) an error occurs due to a shape mismatch.
For case 2, we introduce notation, \shapeN{}, that generates a new dynamic shape.
While the new shape may be identical to other shapes at runtime, our static
analysis is conservative. Each function group has unique shape rules defined below.

\head{Binary element-wise functions.}
Binary element-wise functions take two vectors as input, perform an element-wise operation
and produce a new vector. If either operand is a scalar, the single value is broadcast.
\refTab{rule_elem_binary} presents the shape propagation rules for binary element-wise
functions.

Statically known (constant) vector lengths provide exact shape propagation rules and
can determine error cases at compile time. If one or more argument is a dynamically
known shape, then the resulting shape is also dynamic. If both arguments have the same
unique ID or boolean mask, then the argument shape is propagated. In all other cases,
a new unique shape is generated.

\begin{table}[htbp]
\centering
\caption{Rules for binary element-wise Functions (E)} \label{rule_elem_binary}
\begin{small}
\begin{tabular}{|c||c|c|c|c|}
\hline
$F_{B}$(x,y) & \multicolumn{4}{c|}{x} \\ \hline
y       & \shapeS  & \shapeV{$c_0$} & \shapeV{$d_0$} & \shapeVS{$a_0$} \\ \hline
\shapeS & \shapeS  & \shapeV{$c_0$} & \shapeV{$d_0$} & \shapeVS{$a_0$} \\ 
\shapeV{$c_1$} & \shapeV{$c_1$} & \shapeV{$c_0$}$^1$ & \shapeN & \shapeN \\ 
\shapeV{$d_1$} & \shapeV{$d_1$} & \shapeN & V($d_0$)$^2$ & \shapeN \\ 
\shapeVS{$a_1$}& \shapeVS{$a_1$}& \shapeN & \shapeN & \shapeVS{$a_0$}$^3$ \\ \hline
\multicolumn{5}{|c|}{$^1$: if $c_0$==$c_1$ otherwise error } \\
\multicolumn{5}{|c|}{$^2$: if $d_0$==$d_1$ otherwise \shapeN } \\
\multicolumn{5}{|c|}{$^3$: if $a_0$==$a_1$ otherwise \shapeN } \\
\hline
\end{tabular}
\end{small}
\end{table}

%%% \begin{table}[htbp]
%%% \centering
%%% \caption{Element-wise Functions - Binary} \label{rule_elem_binary}
%%% \begin{tabular}{c||c|c|c|c|c|c|c}
%%% Binary  & \shapeS  & \shapeV{$c_0$} & \shapeV{$c_1$} & \shapeV{$d_0$} & \shapeV{$d_1$} & \shapeVS{$a_0$} & \shapeVS{$a_1$} \\ \hline
%%% \hline
%%% \shapeS & \shapeS  & \shapeV{$c_0$} & \shapeV{$c_1$} & \shapeV{$d_0$} & \shapeV{$d_1$} & \shapeVS{$a_0$} & \shapeVS{$a_1$} \\ \hline
%%% \shapeV{$c_0$} & \shapeV{$c_0$} & \shapeV{$c_0$} & - & \shapeN & \shapeN & \shapeN & \shapeN \\ \hline
%%% \shapeV{$c_1$} & \shapeV{$c_1$} & - & \shapeV{$c_1$} & \shapeN & \shapeN & \shapeN & \shapeN \\ \hline
%%% \shapeV{$d_0$} & \shapeV{$d_0$} & \shapeN & \shapeN & V($d_0$) & \shapeN & \shapeN & \shapeN \\ \hline
%%% \shapeV{$d_1$} & \shapeV{$d_1$} & \shapeN & \shapeN & \shapeN & V($d_1$) & \shapeN & \shapeN \\ \hline
%%% \shapeVS{$a_0$}& \shapeVS{$a_0$} & \shapeN & \shapeN & \shapeN & \shapeN & \shapeVS{$a_0$} & \shapeN \\ \hline
%%% \shapeVS{$a_1$}& \shapeVS{$a_1$} & \shapeN & \shapeN & \shapeN  & \shapeN & \shapeN & \shapeVS{$a_1$} \\ \hline
%%% \end{tabular}
%%% \end{table}

\head{Unary element-wise functions.}
Unary element-wise functions take a single vector as input and produce a new
output vector of the same size. Dataflow rules for shape are the identity in this case.
% \refTab{rule_elem_unary} presents the rules for unary element-wise functions.

\begin{comment}
\begin{table}[htbp]
\centering
\caption{Rules for unary element-wise functions (E)} \label{rule_elem_unary}
\begin{small}
\begin{tabular}{|c||c|c|c|c|}
\hline
$F_{U}(x)$ & \shapeS & \shapeV{$c_0$} & \shapeV{$d_0$} & \shapeVS{$a_0$} \\ \hline
\hline
Return     & \shapeS & \shapeV{$c_0$} & \shapeV{$d_0$} & \shapeVS{$a_0$} \\ \hline
\end{tabular}
\end{small}
\end{table}
\end{comment}

\head{Reduction functions.}
Reduction functions take a vector as input and compute a scalar output value.  In all
cases this thus produces a $V(1)$ shape.
%described in \refTab{rule_reduction}.

\begin{comment}
\begin{table}[htbp]
\centering
\caption{Rules for reduction functions (R)} \label{rule_reduction}
\begin{small}
\begin{tabular}{|c||c|c|c|c|}
\hline
$F_{R}$(x) & \shapeS & \shapeV{$c_0$} & \shapeV{$d_0$} & \shapeVS{$a_0$} \\ \hline
\hline
Return     & \shapeS & \shapeS & \shapeS & \shapeS \\ \hline
\end{tabular}
\end{small}
\end{table}
\end{comment}

\head{Scan function.}
The boolean selection function \texttt{@compress} takes two vectors of equal
length: a boolean mask vector, and a values vector. The output vector contains only
those values with a corresponding \texttt{TRUE} flag in the mask. \refTable{rule_scan}
describes the full shape rules where \texttt{x} is the mask and \texttt{y} the values
vector.

\begin{figure}[htbp]
\begin{lstlisting}[language=HorseIR, frame=single, basicstyle=\footnotesize]
// b:bool, x:i32, y:i32 (vectors of same length)
t0:i32 = @compress(b, x);
t1:i32 = @compress(b, y);
// t0 and t1 share the same scan shape
\end{lstlisting}
\caption{Example propagating the scan shape.} \label{fig:scan_shape}
\end{figure}

For constant sized vectors where the length agrees, the output shape is determined at
runtime. If the lengths differ a compile-time error is thrown. Dynamic length vectors
also generate new scan shapes parameterized on the boolean mask. As multiple value
vectors may be compressed using the same mask, we internally map each boolean mask
to its output shape. When propagating, this map is first checked before generating
a new unique shape. \refFig{fig:scan_shape} shows an example of two vectors which
have the same output scan shape.

%%% \todo{Update table to include ^1 for error cases, V_s missing}

\begin{table}[htbp]
\centering
\caption{Rules for the scan function (S)} \label{rule_scan}
\begin{small}
\begin{tabular}{|c||c|c|c|c|c|}
\hline
$F_{S}$(x,y)   & \multicolumn{4}{c|}{x} \\ \hline
y              & \shapeS & \shapeV{$c_0$} & \shapeV{$d_0$} & \shapeVS{$a_0$} \\ \hline
\shapeS        & \shapeN & \shapeE & \shapeN & \shapeN \\ 
\shapeV{$c_1$} & \shapeE & \shapeN$^1$ & \shapeN & \shapeN \\ 
\shapeV{$d_1$} & \shapeN & \shapeN & \shapeVS{x}$^2$ & \shapeN \\
\shapeVS{$a_1$} & \shapeN & \shapeN & \shapeN & \shapeVS{x} \\ \hline
\multicolumn{5}{|c|}{$^1$: if $c_0$==$c_1$ otherwise error} \\
\multicolumn{5}{|c|}{$^2$: if $d_0$==$d_1$ otherwise \shapeN } \\
\hline
%\multicolumn{6}{c}{Let $t_0$ = \shapeN, then $d_0^{\prime}$ = \shapeVS{$t_0$} if $d_0$==$d_1$ else $t_0$} \\
%\multicolumn{6}{c}{Let $t_1$ = \shapeN, then $a_0^{\prime}$ = \shapeVS{$t_1$} } \\
%\hline
\end{tabular}
\end{small}
\end{table}

%%% \begin{table}[htbp]
%%% \centering
%%% \caption{Rules for the scan function (S)} \label{rule_scan}
%%% \begin{tabular}{c|c||c|c|c|c|c|}
%%% \multicolumn{2}{c}{} & \multicolumn{4}{l}{1st Parameter} \\ \cline{2-6}
%%% \multirow{5}{*}{\rotatebox[origin=c]{90}{2nd Paramter}} &
%%% Scan    & \shapeS & \shapeV{$c_0$} & \shapeV{$d_0$} & \shapeVS{$a_0$} \\ \cline{2-6}
%%% \cline{2-6}
%%% & \shapeS        & \shapeN & \shapeE & \shapeN & \shapeN \\ \cline{2-6}
%%% & \shapeV{$c_1$} & \shapeE & $c_0^{\prime}$ & \shapeN & \shapeN \\ \cline{2-6}
%%% & \shapeV{$d_1$} & \shapeN & \shapeN & $d_0^{\prime}$ & \shapeN \\ \cline{2-6}
%%% & \shapeVS{$a_1$} & \shapeN & \shapeN & \shapeN & $a_0^{\prime}$ \\ \cline{2-6}
%%% %\multicolumn{6}{c}{$c_0^{\prime}$ = \shapeN if $c_0$==$c_1$ else error } \\
%%% %\multicolumn{6}{c}{Let $t_0$ = \shapeN, then $d_0^{\prime}$ = \shapeVS{$t_0$} if $d_0$==$d_1$ else $t_0$} \\
%%% %\multicolumn{6}{c}{Let $t_1$ = \shapeN, then $a_0^{\prime}$ = \shapeVS{$t_1$} } \\
%%% %\hline
%%% \end{tabular}
%%% \end{table}

\head{Array indexing function.}
The array indexing function \texttt{@index} takes two vectors as input (values and
indexes) and performs an indexed read. The output vector therefore contains one element
per index, and thus its shape is determined by the shape of the index vector.
%\refTab{rule_indexing} shows the rules for the array indexing function where \texttt{x}
%contains the values, and \texttt{y} the indexes to fetch.

\begin{comment}
\begin{table}[htbp]
\centering
\caption{Rules for Array Indexing (X)} \label{rule_indexing}
\begin{small}
\begin{tabular}{|c||c|c|c|c|}
\hline
$F_{X}$(x,y) & \multicolumn{4}{c|}{x} \\ \hline
y            & \shapeS & \shapeV{$c_0$} & \shapeV{$d_0$} & \shapeVS{$a_0$} \\ \hline
\shapeS        & \shapeS  & \shapeS  & \shapeS  & \shapeS  \\ 
\shapeV{$c_1$} & \shapeV{$c_1$} & \shapeV{$c_1$} & \shapeV{$c_1$} & \shapeV{$c_1$} \\ 
\shapeV{$d_1$} & \shapeV{$d_1$} & \shapeV{$d_1$} & \shapeV{$d_1$} & \shapeV{$d_1$} \\ 
\shapeVS{$a_1$} & \shapeVS{$a_1$} & \shapeVS{$a_1$} & \shapeVS{$a_1$} & \shapeVS{$a_1$} \\ \hline
\end{tabular}
\end{small}
\end{table}
\end{comment}

\head{Special boolean functions.}
Special boolean functions take a data vector as input and return a boolean vector
indicating adherence to a specified property. For example, \texttt{@like(x, y)} checks
if the data values \texttt{x} match search string \texttt{y}. Functions in this group
therefore return the shape of the first argument.
%Full shape rules are found in
%\refTable{rule_boolean}, where \texttt{x} is the values and \texttt{y} the extra parameter.

\begin{comment}
\begin{table}[htbp]
\centering
\caption{Rules for Special boolean functions (B)} \label{rule_boolean}
\begin{small}
\begin{tabular}{|c||c|c|c|c|}
\hline
$F_{B}$(x,y)  & \multicolumn{4}{c|}{x} \\ \hline
y               & \shapeS & \shapeV{$c_0$} & \shapeV{$d_0$} & \shapeVS{$a_0$} \\ \hline
\shapeS         & \shapeS  & \shapeV{$c_0$} & \shapeV{$d_0$} & \shapeVS{$a_0$} \\ 
\shapeV{$c_1$}  & \shapeS  & \shapeV{$c_0$} & \shapeV{$d_0$} & \shapeVS{$a_0$} \\ 
\shapeV{$d_1$}  & \shapeS  & \shapeV{$c_0$} & \shapeV{$d_0$} & \shapeVS{$a_0$} \\ 
\shapeVS{$a_1$} & \shapeS  & \shapeV{$c_0$} & \shapeV{$d_0$} & \shapeVS{$a_0$} \\ \hline
\end{tabular}
\end{small}
\end{table}
\end{comment}

%%% \begin{table}[htbp]
%%% \centering
%%% \caption{Rules for Special boolean functions (B)} \label{rule_boolean}
%%% \begin{tabular}{c||c|c|c|c|c}
%%% \hline
%%% Boolean  & \shapeS & \shapeV{$c_0$} & \shapeV{$d_0$} & \shapeVS{$a_0$} \\ \hline
%%% \hline
%%% \shapeS        & \shapeS  & \shapeS  & \shapeS  & \shapeS \\ \hline
%%% \shapeV{$c_1$} & \shapeV{$c_0$} & \shapeV{$c_0$} & \shapeV{$c_0$} & \shapeV{$c_0$} \\ \hline
%%% \shapeV{$d_1$} & \shapeV{$d_0$} & \shapeV{$d_0$} & \shapeV{$d_0$} & \shapeV{$d_0$} \\ \hline
%%% \shapeVS{$a_1$} & \shapeVS{$a_0$} & \shapeVS{$a_0$} & \shapeVS{$a_0$} & \shapeVS{$a_0$} \\ \hline
%%% \end{tabular}
%%% \end{table}

\subsubsection{Lists}

List functions apply a function on cells individually and merge the results into a new list.
For example, the \texttt{@each} function is shown in \refFig{fig:list_function}. In the
following sections we denote the applied function as \texttt{@f}.

\begin{figure}[htbp]
\begin{lstlisting}[language=HorseIR, frame=single, basicstyle=\footnotesize]
// x:i32, y:i32 vectors; @f function
t0:list<i32> = @list(x, y);
t1:list<i32> = @each(@f, t0);
// t1 contains cells [@f(x), @f(y)]
\end{lstlisting}
\caption{Example of a list function.} \label{fig:list_function}
\end{figure}

\sloppy % avoid a math line to be too long
\head{Function each} applies function \texttt{@f} to each cell in a list. Each cell shape
is therefore transformed by the function to create a new list. Given input shape
\texttt{list<$L_0, \ldots, L_{n-1}$>} the output shape is thus
\texttt{list<@f($L_0\texttt{)}, \ldots, \texttt{@f(}L_{n-1}\texttt{)}$>}.
 
\head{Function each\_left} takes two parameters: a list and a variable \textit{of any type}.
The function is applied on each cell of the list and the variable to form a new list with cells for
each pairing. Given input shapes \texttt{list<$L_0, \ldots, L_{n-1}$>} and \texttt{A}, the function
produces a new list with shape \texttt{list<@f($L_0\texttt{, A)}, \ldots, \texttt{@f(}L_{n-1}\texttt{, A)}$>}.

\head{Function each\_right} takes two parameters: a variable \textit{of any type} and a list.
The function is applied on the variable and each cell of the list to form a new list with cells for
each pairing. Given input shapes \texttt{A} and \texttt{list<$L_0, \ldots, L_{n-1}$>}, the function
produces a new list with shape \texttt{list<@f($\texttt{A, }L_0\texttt{)}, \ldots, \texttt{@f(A, }L_{n-1}$)>}.

\head{Function each\_item} takes two lists of equal length as input and evaluates the given
function on each pair of cells to form a new list. Given input shapes \texttt{list<$L_a$>} and
\texttt{list<$L_b$>} the function returns a new list of shape
\texttt{list<@f($L_{a0}, L_{b0}\texttt{)}, \ldots, \texttt{@f(}L_{a(n-1)}, L_{b(n-1)}\texttt{)}$>}.

\head{Function raze} flattens a homogeneous list of vectors into a single vector, removing cell
divisions. For any list the output shape is a dynamically sized vector \shapeV{d}.

\subsubsection{Other Functions}

For all other functions, a new dynamic shape (either list or vector depending on the return type) is 
generated as the output shape. This is conservative, but correctly prevents fusing any unknown or
non-fusible function. Further optimization is possible using pre-defined patterns.
