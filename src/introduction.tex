While database systems have traditionally stored relational tables on a per-row
basis, recent developments have shown that a column-store format can outperform
traditional RDBMS for many workloads, particularly when the data fits into
main memory.
One of the first research prototypes of a column-based database manage systems
(DBMS) was MonetDB \cite{IdreosS2012}, still an active research project, but
many commercial systems now also support column-based storage (e.g.
Microsoft SQL Server~\cite{msqlserver},
SAP HANA~\cite{FarberF2012}, and
HyPer~\cite{Neumann2011:HyPer}).
% Microsoft SQL server
This fundamental shift in database architecture has resulted in a large body of
research efforts within the database community looking into efficient SQL query
execution on such a data format.

Given that in a column store each column of a table is stored as an array,
array programming languages appear as a promising construct to provide support
for query execution. In fact, \OldPaperAuthor introduced an intermediate
representation (IR) language, called \textit{HorseIR}, targeted at database
support.  SQL queries are first transformed into execution plans using
standard database optimization techniques that describe in which order and
how database operators such as joins, selections and aggregations should be
executed depending on the data to be analyzed. These execution plans are
then translated into HorseIR, the resulting code is optimized using
techniques from array programming, and finally the optimized code is
compiled into C code executable on different platforms.
% \OldPaper shows very good results on standard database benchmarks.

The approach used in HorseIR, however, does not take full advantage of the optimization techniques available. Instead, it is restricted to basic code fusion of element-wise operators, and although it explores some high-level pattern-based optimization techniques, these require considerable knowledge of database operators.  Optimization potential is also reduced by extensive use of built-in functions, the complexity of which limits optimization.

For example, it is common for SQL queries to do a selection (selecting some rows based on the filter criteria on one column) and then perform aggregation, such as a summation.  In an array-based language this can be represented as

\begin{small}
\begin{Verbatim}[xleftmargin=.3\columnwidth]
R = sum(filter(C, A))
\end{Verbatim}
\end{small}

%%% $$
%%% R = sum(filter(C, A))
%%% $$

\noindent{}where the condition \texttt{C} is a boolean vector, the input array \texttt{A}
is a vector with the same number of elements, and the result \texttt{R} is a
scalar.  A straightforward way to generate C code from this is to generate two loops, the first applying the boolean selection and producing an intermediate result, and the second loop performing the reduction on this intermediate result.  A more optimized version, however, would fuse the two loops into one, performing the reduction on the fly and avoiding generation of an intermediate result:

% add lines around box: frame=single
\begin{small}
\begin{Verbatim}[xleftmargin=.15\columnwidth]
R=0;
for(int i=0; i<C.size; i++) {
    if(C[i]) { // true block
        R += A[i];
    }
}
\end{Verbatim}
\end{small}

%%% \begin{lstlisting}[style=CStyle]
%%% for(int i=0; i<10; i++){
%%%     if(C[i]) { // true block
%%%         R += A[i]
%%%     }
%%% }
%%% \end{lstlisting}

%% However, HorseIR describes both sum and filter as built-in functions, and
%% currently no optimization across those functions can be performed.
\noindent{}The analysis required to generate such code from a series of built-in functions, however, is complex,
depending on non-trivial array dependency analysis as well as knowledge of array dimensions, and 
while possible in principle, our experience has been that these optimization opportunities are not found by our available C compilers.

In this paper we provide a more systematic approach to
analyzing the query program and generating optimized code. In particular, we provide
a detailed analysis of the most important built-in functions that are useful
for executing SQL operators, and how we can optimize across such functions.
To do so, we have a closer look at \textit{shape analysis}, and how shape propagates
when functions are combined. This allows us to use loop fusion extensively to
avoid unnecessary iterations over the data and intermediate results.  

% contribution
The contributions of this paper are summarized as follows:

\begin{itemize}
\item We explore a new approach to improve database query performance by
      applying techniques derived from array programming. We focus on HorseIR, an IR specifically designed to support database query execution.
\item We perform shape analysis on the built-in functions of HorseIR that represent database operators. This allows us to collect precise shape information and provide a conformability analysis to identify fusible sections in the HorseIR code. 
\item We present a set of optimization and code generation strategies to automatically
      generate optimized code.
\item We conduct experiments on the database benchmark TPC-H, to show the effectiveness of our technique.
\end{itemize}

\begin{comment}
The rest of this paper is organized as follows:
\refSec{Sec:background}
provides the background of HorseIR and array-based optimizations.
\refSec{Sec:overview}
introduces the description of the overview of our approach.
\refSec{Sec:shape}
presents our shape analysis for different groups of functions.
\refSec{Sec:conformability}
introduces our conformability analysis to identify fusible sections.
\refSec{Sec:codegen}
describes our code generation rules for generating efficient code.
\refSec{Sec:evaluation}
shows our experiments and experimental results.
\refSec{Sec:relatedwork}
discusses related work, and
\refSec{Sec:conclusion}
concludes.
\end{comment}