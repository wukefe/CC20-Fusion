% \subsection{Fusion Teachniques}
% \subsection{Related Database Systems}

Fusion techniques are popular due to their success in reducing intermediate
results. This is especially true for compiled environments which easily
allow code fusion.

In the MATLAB-to-FORTRAN compiler~\cite{rose1999:techniques}, shape and size
information can be obtained from a conformability analysis with a set of
well-defined conformable operators for scalar, vector, and matrix.
However, no further operator fusions after conformability analysis are provided.

In the R programming language, a vectorizer is proposed for the built-in
function \texttt{Apply}~\cite{Wang2015:vectorization}. The function
\texttt{Apply} is similar to our list-based functions, which take
a function and a list of data inputs as parameters and repeatedly apply
the function. Their vectorizer involves both code and data transformation
while we focus on code optimization. Additionally, we explore a wider set
of built-in functions rather than only a single list-based function.

Ju et al.~\cite{ju1994:array} investigate built-in function fusion in a pure
array-based programming language, APL. They classify operators into categories
based on the features of the functions and present a model to help generate
parallel code when fusing built-in functions. Ching et al.~\cite{ChingZ12parallel}
use simple techniques to fuse arithmetic functions when compiling APL to parallel
C code. In contrast, our work is aimed at array-based programs generated from
SQL queries, considering functions important to database operators.

HyPer~\cite{Neumann2011:HyPer}, an in-memory database system, adopts a data-centric
model when compiling SQL queries to LLVM code. Their model combines algebraic
operators in database execution plans to avoid generating intermediate results.
HorseQC~\cite{FunkeBNMT18:HorseQC} proposes \textit{fusion operators} for
fusing multiple relational operations. Both HyPer and HorseQC share our goal
of fusing operators. However, we approach the problem from a static analysis
perspective that operates on the IR-level rather than directly on the execution plan.

TVM~\cite{Chen18:TVM}, a system designed for deep learning, introduces
operator fusion for graph operators when generating efficient GPU code.
Their approach is limited, as the fusion rules are fairly simple and
specific to fusion between categories.  We provide a more sophisticated
fusion approach which considers more categories and defines systematic
data-flow analysis to identify fusion opportunities.

Similar to optimizations for arrays~\cite{KumarH14,FoleyH16matjuice},
we exploit type and shape information of arrays to generate efficient code.
But due to the high-level semantics of HorseIR programs, we avoid
complicated vectorization techniques~\cite{menon1999case,ChenKLH16}.

When loop fusion is applied to query related code, it is complicated to
identify optimization opportunities by using a predefined set of rules.
The DBLAB/LB query compiler~\cite{ShaikhhaKPBD016} 
provides fusion rules for different loop fusion algorithms to generate optimized code
\cite{Shaikhha2018:LoopFusion}, however, just as the patterns in \OldPaper, it is
challenging to generalize. % page 21
The Peloton in-memory DBMS~\cite{Menon2017:Peloton} considers operator
fusion for operators within a pipeline. Again, the use quite specialized fusion
rules, while we provide a more general optimization approach. 

% \todo{Q: Do we need to mention some papers/tools/systems which do not support
% operator fusion? for example, MonetDB,HyPer,...}


