% packages
\usepackage{listings}
\usepackage{graphicx}
\usepackage{fancyhdr}
\usepackage{fancyvrb} % for Verbatim
\usepackage{hyperref}
\usepackage{color}
\usepackage{import}
\usepackage[ruled,vlined,noend]{algorithm2e} %see algorithm2e.pdf
\usepackage{url}
\usepackage[framemethod=TikZ]{mdframed}
%\usepackage{amsmath}
\usepackage{amsfonts} % for checkmark
\usepackage{multirow}
\usepackage{flexisym}
\usepackage{wrapfig}
\usepackage{tikz} % draw trees
\usepackage{epstopdf}
\usepackage{seqsplit}
\usepackage{graphics}

% For strike through of font - Joseph
\usepackage[normalem]{ulem}
%\usepackage{cancel}
%\usepackage{soul}
% For inparaenum environment. - Joseph
\usepackage{paralist}
\usepackage{comment}
\usepackage{setspace} % setstretch
\usepackage{svg}

% names
\newcommand{\addr}{School of Computer Science, McGill University, Montr\'eal, Canada}

% colors
\definecolor{Orange}{rgb}{1,0.5,0}
\definecolor{Blue}{RGB}{0,0,255}
\definecolor{OrangeRed}{RGB}{255,69,0}
\definecolor{LightGreen}{RGB}{11,83,69}
\definecolor{DarkYellow}{RGB}{160,64,0}
\definecolor{AntiqueBrass}{rgb}{0.8, 0.58, 0.46}
\definecolor{NiceBlue}{RGB}{11, 102, 163}
\definecolor{NiceRed}{RGB}{160, 64, 0}
\definecolor{Grey}{RGB}{128,128,128}
\definecolor{SlightRed}{RGB}{249,38,114}
\definecolor{SlightPink}{RGB}{155, 89, 182}

% I am using this color to highlight the text I am writing so that it is easy for you - Joseph
\definecolor{PineGreen}{rgb}{0.0, 0.47, 0.44}

\newcommand{\hgb}[1]{{\emph{\color{Blue}{#1}}}}
% \newcommand{\hgr}[1]{{\emph{\color{OrangeRed}{#1}}}}
\newcommand{\hgr}[1]{{\emph{\color{NiceRed}{#1}}}}
\newcommand{\hg}[1]{\emph{#1}}
\newcommand{\hglg}[1]{\emph{{\color{LightGreen}{#1}}}}
\newcommand{\hgdy}[1]{\emph{{\color{DarkYellow}{#1}}}}
\newcommand{\hga}[1]{\emph{{\color{AntiqueBrass}{#1}}}}

% tools
\newcommand{\todo}[1]{\hgb{Todo: #1}}
\newcommand{\joseph}[1]{\hglg{Joseph: #1}}
\newcommand{\hongji}[1]{\hgdy{Hongji: #1}}
\newcommand{\hanfeng}[1]{\hgr{Hanfeng: #1}}

% edit
\newcommand{\head}[1]{\vspace{1mm}\noindent\textbf{#1}\hspace{2mm}}
\newcommand{\headstep}[1]{\vspace{1mm}\noindent\textit{#1}\hspace{2mm}}
%\renewcommand{\baselinestretch}{1.2}
%\renewcommand{\lstlistingname}{Code} % update listing name's prefix

% ref
\newcommand{\refSec}[1]{Sec.~\ref{#1}}
\newcommand{\refFig}[1]{Fig.~\ref{#1}}
\newcommand{\refTable}[1]{Table~\ref{#1}}
\newcommand{\refListing}[1]{Listing~\ref{#1}}
\newcommand{\refAlgo}[1]{Algo.~\ref{#1}}
\newcommand{\refFoot}[1]{\footnote{\footnotesize{#1}}}

% syntax highlighting
\usepackage{textcomp} % other glyphs needed for upquote in listings below
\lstdefinelanguage{HorseIR}
{ basicstyle=\footnotesize\ttfamily,
  commentstyle=\color{Grey}\rmfamily\itshape,
  keywordstyle=\color{SlightPink},
  keywordstyle=[2]\color{NiceBlue},
  keywordstyle=[3]\color{NiceRed},
  keywordstyle=[4]\color{SlightRed},
  % list of keywords
  morekeywords={
  module,
  def,
  import
  },
  keywords=[2]{
  @table,
  @sum,
  @gt,
  @load_table,
  @column_value,
  @and,
  @compress,
  @enum,
  @list,
  @group,
  @values,
  @each_right,
  @raze,
  @compute_discount,
  @geq,
  @lt,
  @each,
  @len,
  @vector,
  @index_a,
  @index,
  @unique,
  @mul
  },
  keywords=[3]{
  table,
  sym,
  list,
  i64,
  f64,
  bool,
  },
  keywords=[4]{
  return,
  check_cast
  },
  sensitive=false, % keywords are not case-sensitive
  morecomment=[l]{//}, % l is for line comment
  morecomment=[s]{/*}{*/}, % s is for start and end delimiter
  morestring=[b]", % defines that strings are enclosed in double quotes
  upquote=true % ensures that backtick displays correctly
}

% \usepackage[subtle]{savetrees}
%\usepackage{flushend} % has balanced columns on the last page
\usepackage{balance} % has balanced columns on the last page

%%%%%%%%%%%%%%%%%%%%%%%%%%%%%%%%%%%
%%% below for Damon 2019        %%%
%%%%%%%%%%%%%%%%%%%%%%%%%%%%%%%%%%%

\definecolor{mGreen}{rgb}{0,0.6,0}
\definecolor{mGray}{rgb}{0.5,0.5,0.5}
\definecolor{mPurple}{rgb}{0.58,0,0.82}
\definecolor{backgroundColour}{rgb}{0.95,0.95,0.92}

\lstdefinestyle{CStyle}{
   % backgroundcolor=\color{backgroundColour},   
   % commentstyle=\color{mGreen},
    commentstyle=\color{Grey}\rmfamily\itshape,
    %keywordstyle=\color{magenta},
    keywordstyle=\color{Blue},
    %numberstyle=\tiny\color{mGray},
    numberstyle=\color{mGray},
    stringstyle=\color{mPurple},
    %basicstyle=\footnotesize,
    basicstyle=\footnotesize\ttfamily,
    breakatwhitespace=false,         
    breaklines=true,                 
    captionpos=b,                    
    keepspaces=true,                 
    numbers=left,                    
    numbersep=5pt,                  
    showspaces=false,                
    showstringspaces=false,
    showtabs=false,                  
    tabsize=2,
    language=C,
    frame=single,
    xleftmargin=2mm,
    xrightmargin=2mm
}

\usepackage{xspace}
\usepackage{subcaption}

\newcommand{\scan}[2]{$\sigma$(#1, #2)}
\newcommand{\tuple}[6]{$\sum$(#1, #2, #3, #4, #5, #6)}
%\newcommand{\tupleSingle}[5]{$\sum$(#1, #2, #3, #4, #5)}
\newcommand{\TupleId}[2]{$\Gamma$(#1, #2)}
\newcommand{\TupleEOne}[1]{\TupleId{$E_1$}{#1}}
\newcommand{\TupleETwo}[1]{\TupleId{$E_2$}{#1}}
\newcommand{\TupleCase}[2]{$\Gamma\{$#1, #2$\}$}
\newcommand{\refTab}[1]{Table \ref{#1}}
\newcommand{\vv}{\vspace{3mm}}

\newcommand{\ruleFail}{\textit{NC}\xspace}  % NC = Not Conformable
\newcommand{\tupleNew}[3]{$\langle$#1, #2, #3$\rangle$}
\newcommand{\scanNew}[1]{$\sigma$(#1)}


\newcommand{\OldPaper}{\cite{Chen2018:HorseIR}\xspace}
\newcommand{\OldPaperAuthor}{Chen et al. \OldPaper}
\newcommand{\codeNCC}{\textit{NCC}\xspace}
\newcommand{\codeNCA}{\textit{NCA}\xspace}

\usetikzlibrary{shapes,snakes}


\lstdefinestyle{SQLStyle}{
   % backgroundcolor=\color{backgroundColour},   
    commentstyle=\color{mGreen},
    %keywordstyle=\color{magenta},
    keywordstyle=\color{mPurple},
    %numberstyle=\tiny\color{mGray},
    numberstyle=\color{mGray},
    %stringstyle=\color{mPurple},
    %basicstyle=\footnotesize,
    basicstyle=\footnotesize,
    breakatwhitespace=false,         
    breaklines=true,                 
    captionpos=b,                    
    keepspaces=true,                 
    numbers=left,                    
    numbersep=5pt,                  
    showspaces=false,                
    showstringspaces=false,
    showtabs=false,                  
    tabsize=2,
    language=sql,
    frame=single,
    xleftmargin=2mm
}

%%%%%%%%%%%%%%%%%%%%%%%%%%%%%%%%%%%
%%% below for CGO 20            %%%
%%%%%%%%%%%%%%%%%%%%%%%%%%%%%%%%%%%

\newcommand{\symbot}{$\bot$\xspace}
\newcommand{\symtop}{$\top$\xspace}
\newcommand{\symscan}[1]{$\sigma$(#1)}
\newcommand{\tupleSingle}[5]{\{(#1, #2, #3, #4, #5)\}}

\newcommand{\shapeS}{V(1)}
\newcommand{\shapeU}{U}
% \newcommand{\shapeVD}[1]{V$_d$(#1)}
% \newcommand{\shapeVC}[1]{V$_c$(#1)}
\newcommand{\shapeV}[1]{V(#1)}
\newcommand{\shapeVS}[1]{V$_s$(#1)}
\newcommand{\shapeN}{$I$}
\newcommand{\shapeE}{error}
\newcommand{\assign}{$\gets$}
\newcommand{\notok}{$\times$}
\newcommand{\pass}{\checkmark\xspace}

\newcommand{\shapeL}[2]{L(#1,#2)}
\newcommand{\rulesep}{\unskip\ \vrule\ } % add a vertical line between subfigure
